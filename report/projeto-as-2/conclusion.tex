\chapter{Conclusão}

\hspace{5mm} Em suma, conclui-se que os objetivos inicialmente propostos foram atingidos. Desta forma, conseguiu-se identificar todos os tipos de \textit{code smells} e aplicar as técnicas de \textit{refactoring} adequadas. 

\hspace{5mm} A análise inicial do código fornecido, foi complicada, pois a arquitetura do mesmo, não estava correta, não sendo perceptível algumas das decisões tomadas pelos desenvolvedores do mesmo, quanto à utilização de hierarquia. Do mesmo modo, também percebeu-se que alguns dos \textit{design patterns}, não estavam bem aplicados.

\hspace{5mm} Uma das principais conclusões a que se chegou, foi que, devido à elevada identificação de \textit{code smells}, provavelmente o \textit{refactoring} seria mais dispendioso a nível de tempo, do que, fazer a aplicação de novo, pois existem muitas dependências no código fornecido, tornando difícil a sua alteração.

\hspace{5mm} Finalizado este trabalho, percebe-se a importância da identificação destes \textit{code smells}, sendo crucial, após o seu conhecimento, tentar evitar que eles ocorram nos nossos códigos, para prevenir os problemas que os mesmos causam. 

