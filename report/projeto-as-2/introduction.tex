\chapter{Introdução}

\hspace{5mm} Na unidade curricular de Arquiteturas de Software, foi-nos proposta a identificação de \emph{code smells} e \emph{refactoring} dos mesmos, num código fornecido pela equipa docente.

\hspace{5mm} Na verdade, o código fornecido pela equipa docente, foi desenvolvido por colegas do curso, para o primeiro projeto da unidade curricular, consistindo numa plataforma de \emph{Trading}.

\hspace{5mm} Ao longo deste relatório, aborda-se os tipos de \emph{code smells} encontrados, e os tratamentos aplicados aos mesmos, e as respetivas técnicas de \emph{refactoring} utilizadas. Ainda são determinadas e calculadas algumas métricas que inflênciam a estrutura do código.

\hspace{5mm} O objetivo principal deste projeto, consiste na perceção dos problemas que existem com o desenvolvimento de código, sendo estes responsáveis por dificuldades encontradas nos processos de análise, debugging e adição de novas funcionalidades à aplicação.

\hspace{5mm} Por fim, este projeto também se torna importante, para se refletir dos \emph{code smells} também cometidos no primeiro projeto desenvolvido pelo grupo, nesta unidade curricular. Sendo assim, o conhecimento destes \emph{code smells} é necessário, para não serem novamente cometidos.